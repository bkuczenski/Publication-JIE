\def\equalskip{14pt}

\newgray{matrixgray}{0.22}

\newpsstyle{matrix}{linewidth=0.8pt,linecolor=matrixgray,framesep=0pt}
\newpsstyle{imatrix}{style=matrix,linestyle=dotted}
\newpsstyle{brace}{braceWidth=0.8pt,braceWidthInner=2pt,braceWidthOuter=2pt,linecolor=darkgray}
\newpsstyle{timesline}{linewidth=0.6pt,linecolor=gray,linestyle=dashed,dash=3pt 2pt}

\newpsstyle{vline}{linestyle=solid,linewidth=0.6pt,linecolor=darkgray}

\newpsstyle{upline}{linestyle=dashed,linecolor=black,dash=6pt 0.8pt,linewidth=0.6pt}

\newpsstyle{refplot}{framearc=0.1,linestyle=dashed,fillstyle=none,linecolor=gray, dash=3.3pt 2pt}
\newpsstyle{exchplot}{framearc=0.15,linestyle=solid,linewidth=0.6pt,fillstyle=solid,fillcolor=gray,opacity=0.15}
\newpsstyle{upr}{framearc=0.15,linestyle=none,linewidth=1.6pt,fillstyle=solid,fillcolor=blue,opacity=0.65}
\newpsstyle{foreground}{linewidth=2pt,linecolor=blue,strokeopacity=0.8,fillstyle=solid,fillcolor=steelblue,opacity=0.15}


\newcommand{\newequal}[3][$=$]{%
  %% #1 - operator, default $=$
  %% #2 - left hand side of operator
  %% #3 - nodename for right hand side of operator
  \pnode[\equalskip,0](#2){pOp}
  \pnode[\equalskip,0](pOp){#3}
  \rput(pOp){#1}
  \rput[B](pOp|pEquation){#1}
}

\def\largeequal#1#2{%
  %% #1 - left hand side of equal
  %% #2 - nodename for right hand side of equal
  \pnode[\equalskip,0](#1){pEq}
  \pnode[\equalskip,0](pEq){#2}
  \rput[B](pEq){\Large $=$}
}

\def\matrixtimes#1#2#3{%
  %% #1 - top-facing node - multiplier
  %% #2 - left-facing node - multiplicand
  %% #3 - name of times node
  \pnode(#1|#2){timesnode}
  \rput(timesnode){\rnode{#3}{\large $\times$}}
  \pcline[style=timesline](#1)([angle=-90,nodesep=3pt]#3)
  \pcline[style=timesline](#2)([angle=0,nodesep=3pt]#3)
}

\newcommand{\newmatrix}[4][]{%
  %% #1 - box contents (default empty)
  %% #2 - node name
  %% #3 - height dimension
  %% #4 - width dimension
  \rnode{#2}{\psframebox[style=matrix]{\rule{0pt}{#3}\rule{#4}{0pt}}}
  \rput(#2){#1}
}

\def\plotframe#1{
  %% #1 - style (mandatory)
  \psframe[style=#1](-0.35,-0.35)(0.35,0.35)
}

\newcommand{\fillmatrix}[2][upr]{
  %% make it blue (upr)
  %% #1 style #2 node
  \psframe[style=#1]([angle=180]#2|[angle=-90]#2)([angle=0]#2|[angle=90]#2)
}

\newcommand{\plotmatrix}[5][1pt]{
  %% #1 - xy dimension
  %% #2 - thing to plot
  %% #3 - matrix rnode
  %% #4 - i(row)
  %% #5 - j(col)
  \rput([angle=180]#3|[angle=90]#3){
    \psset{unit=#1}
    \rput(#5,-#4){
      \rput(-0.5,0.5){#2}
    }
  }
}

\newcommand{\matrixvline}[3][style=vline]{
  %% #1 - style
  %% #2 - matrix node name
  %% #3 - x position
  \rput([angle=180]#2){
    \psset{unit=\coordstep}
    \rput(#3,0){
      \psline[#1](0,0|[angle=-90]#2)(0,0|[angle=90]#2)
    }
  }
}
  
\newcommand{\matrixhline}[3][style=vline]{
  %% #1 - style
  %% #2 - matrix node name
  %% #3 - x position
  \rput([angle=90]#2){
    \psset{unit=\coordstep}
    \psline[#1]([angle=180]#2|0,-#3)([angle=0]#2|0,-#3)
  }
}
  

\newcommand{\canonicalref}[3][$\tilde{y}$]{%
  %% #1 - label (\tilde{y})
  %% #2 - matrix
  %% #3 - xy dimension
  \plotmatrix[#3]{$\ast$}{#2}{1}{1}
  \rput([angle=0]#2|[angle=90]#2){%
    \psset{yunit=#3}
    \rput(0,-0.5){
      \rput[l](3pt,0){#1}
    }
  }
}

\newcommand{\colindex}[4][1]{
  %% #1 - index step
  %% #2 - node name
  %% #3 - step size
  %% #4 - number of steps
  \rput([angle=180]#2|[angle=90]#2){%
    \psset{xunit=#3}
    \multido{\i=1+#1}{#4}{%
      \rput(\i,0){
        \rput[b](-0.5,3pt){\i}
      }
    }
  }
}

\newcommand{\refs}[3]{
  %% #1 - node name
  %% #2 - step size
  %% #3 - number of steps
  \multido{\i=1+1}{#3}{%
    \plotmatrix[#2]{\plotframe{refplot}}{#1}{\i}{\i}
  }
}


\def\newImatrix#1#2{%
  %% #1 - node name
  %% #2 - dimension
  \rnode{#1}{\psframebox[style=imatrix]{\rule{0pt}{#2}\rule{#2}{0pt}}}
  \rput(#1){$I$}
}

\def\invertmatrix#1{%
  \rput[Bl]([angle=0,nodesep=2pt]#1|[angle=90]#1){$-1$}
}

\def\leftbrace#1#2#3{%
  %% #1 - node to be enclosed by braces
  %% #2 - text
  %% #3 - name of node
  \pnode([angle=180,nodesep=3pt]#1|[angle=90,nodesep=0]#1){brace01}
  \pnode([angle=180,nodesep=3pt]#1|[angle=-90,nodesep=0]#1){brace02}
  \psbrace*[style=brace,ref=rc,rot=180,nodesepA=-6pt](brace01)(brace02){%
    \pnode(0,0){#3}}
  \rput(#3){#2}
}

\def\topbrace#1#2#3{%
  %% #1 - node to be enclosed by braces
  %% #2 - text
  %% #3 - name of node
  \pnode([angle=0]#1|[angle=90,nodesep=3pt]#1){brace01}
  \pnode([angle=180]#1|[angle=90,nodesep=3pt]#1){brace02}
  \psbrace*[style=brace,ref=bC,rot=-90,nodesepB=-3pt](brace01)(brace02){%
    \pnode(0,0){#3}}
  \rput[Br]([angle=0,nodesep=12pt]#3){#2}
}

\def\toplabel#1#2{%
  %% #1 - matrix
  %% #2 - label
  \rput[b]([angle=90,nodesep=3pt]#1){#2}
}

\def\bottomlabel#1#2{%
  %% #1 - matrix
  %% #2 - label
  \rput[t]([angle=-90,nodesep=3pt]#1){#2}
}

\def\leftlabel#1#2{%
  %% #1 - matrix
  %% #2 - label
  \rput[r]([angle=180,nodesep=3pt]#1){#2}
}

%%%%%%%%%%%%%%%%%%%%%%%%%%%%%%%%%%%%%%%%%%%%%%%%%%
%% auto coord-drawers
%% These require the nodes to be named n# where # is an index argument
%% They also require the macro \coordstep to be defined as the index size in the matrix

\def\cnode(#1)#2{
  %% #1- coords
  %% #2- index
  \pnode(#1){p#2}
  \rput(p#2){\circlenode{n#2}{#2}}
}

\def\bnode(#1)#2{
  %% #1- coords
  %% #2- index
  \pnode(#1){p#2}
  \rput(p#2){\rnode{n#2}{#2}}
}

\def\connect#1(#2,#3)#4{
  %% #1 - node name
  %% #2 - row / i
  %% #3 - col / j
  %% #4 - what to draw in the box
  \plotmatrix[\coordstep]{#4}{#1}{#2}{#3}
  \ncline[style=standard,nodesep=2pt]{->}{n#2}{n#3}
}

\def\labelconnect[#1]#2(#3,#4)#5{
  %% #1 - label
  %% #2 - node name
  %% #3 - row / i
  %% #4 - col / j
  %% #5 - what to draw in the box
  \connect{#2}(#3,#4){#5}
  \aput[2pt](0.4){#1}
  \plotmatrix[\coordstep]{#1}{#2}{#3}{#4}
}
