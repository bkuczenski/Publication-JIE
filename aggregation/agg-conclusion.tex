\section{Conclusion} 

The current practice of documenting a PSM with a written report sharply limits the ability of readers to interpret and reuse the data.  It also results in the model being frozen to a particular configuration or set of configurations used to generate the published results.  Any review objectives involving sensitivity of the model to parameter variations and/or alternative scenarios are limited to the selections considered by the author.
When a study result is published in an aggregated form, often it is beyond the capacity of the critical reviewer to validate the aggregation result, even if the reviewer has privileged access to confidential materials used in preparing the report. 

This paper presents a possible solution by providing a mathematical formalization of the foreground of an LCA study and a functional specification for disclosing the PSM.  The non-quantitative portions of the disclosure make up a formal definition of study scope and the boundary of the foreground system.  The disclosure framework also allows modelers to express precisely how multi-functionality is handled through allocation or system expansion.  A data user or critical reviewer with access to the six disclosure elements, and also adequate access to background datasets and elementary flows used in the model, can reproduce the LCA computation in Eq.~\ref{eqn:study}, and ensure the reported result is correct.  Once equipped with the PSM, the reviewer or data user can potentially go much further.  Practitioners who are able to reproduce the PSM from the disclosure can also modify it, adding or removing elements, altering dataset selections, applying other impact assessment methods, and considering alternative allocation methods. Portions of the model which have been pre-aggregated can remain private in the disclosure. Further research is required to determine whether or under what conditions aggregation results can be reverse-engineered.

If data providers also make available stable semantic references to background data sets and elementary flow characterizations, then it becomes possible for a stand-alone description of a PSM to be used to reproduce an LCA result.  The same description can be extended or incorporated into subsequent studies by other authors. From this prescription, a framework for distributed, platform-independent LCA computation can be imagined.
