\subsection{Foreground Study}

In practice, most LCA studies contain foreground information that is not present in a background database, and LCA software systems generally allow users to augment the standard LCIDBs with foreground models.  To meet our disclosure objectives, it is important to be able to distinguish foreground content from background content in the computation.  Background databases are prepared independently of any study and do not typically vary across the studies in which they are used.  However, the foreground systems themselves cannot be computed without the background.  Put another way, the foreground depends on the background but the background does not depend on the foreground.  This condition allows the processes in an LCIDB to be partially ordered to group foreground processes together \citep{Kuczenski_JLCA_2015}.  

The simplest conceivable foreground is simply the final demand $\mathbf{y}$, in other words the list of background processes invoked by the study and their activity levels.  For an LCA computation as expressed in Eq.~\ref{eqn:leontief}, the same result can be obtained by relocating the $\mathbf{y}$ vector into an augmented $A$ matrix, augmenting the $B$ matrix with a vector of zeros, and replacing the final demand with a vector $\tilde{\mathbf{y}}$ in which the first entry is 1 and all successive entries are 0:
\begin{equation}
  \begin{array}{cc}
s =& \mathbf{e}^T\times B\times\left(I-A\right)^{-1}\times\mathbf{y} \\
 =& \mathbf{e}^{T}\times\tilde{B}\times\left(I-\tilde{A}\right)^{-1}\times\tilde{\mathbf{y}}
  \end{array}
  \label{eqn:canonical}
\end{equation}
where $\tilde{A} = \left[\begin{smallmatrix}
 0 & \mathbf{0} \\
 \mathbf{y} & A
  \end{smallmatrix}\right]$, $\tilde{B} = [ \mathbf{0},\, B ]$, and $\tilde{\mathbf{y}} =  [ 1,\, 0,\, 0 ,\,\ldots,\, 0]^{T}$.
It can easily be shown that Eq.~\ref{eqn:canonical} holds for any $\mathbf{y}$.  \textit{author's note: a proof for this equality is provided in the electronic supplementary information.}  This is intuitively true because the $\tilde{\mathbf{y}}$ selects the first column of the augmented $\tilde{A}$ matrix as the ``final demand,'' but the ``direct requirements'' of that column are precisely the final demand from the prior, un-augmented $A$ matrix.  

\begin{figure}
  \begin{center}
    \input{figures/fig1-foreground.fig}
    \caption{Matrix structure of an LCA foreground study.  The background technology matrix $X$ has been replaced with a background LCI database $B_x$.  The foreground input matrices and aggregation results are highlighted.}
    \label{fig:foreground}
  \end{center}
\end{figure}

Of course, most studies have a more elaborate foreground than simply the demand for processes in the background database, and in these cases the $A$ and $B$ matrices can be further augmented with additional foreground content.  An LCA study augmented with foreground information can be written in block-triangular form:
\begin{equation}
\tilde{A} = \left[\begin{array}{cc}
A_f & 0 \\
A_d &  A
  \end{array}
\right];\;\;\;
  \tilde{B} = \left[\begin{array}{cc} B_f & B   \end{array}\right]
\label{eqn:foreground}
\end{equation}
Here, the submatrix $A_f$ represents the foreground (the first column of which delivers the functional unit of the study); $A$ represents the background, and the rectangular matrix $A_d$ represents the dependency of the foreground on the background.  The ordered $\tilde{B}$ matrix is similarly partitioned into $B_f$, which includes foreground emissions, and $B$ which includes background emissions. The constitutive characteristic of this formulation is that the submatrix in the top right corner of $\tilde{A}$ is zero.  As long as this is the case, all computations regarding the background are invariant with respect to any foreground, and they can be computed in advance. However, it does not apply for so-called ``integrated hybrid'' models, in which all quadrants contain non-zero cells (for instance, \cite{Gibon_2015}).

%For clarity, I will call thein which the foreground content always appears to the left of the background content; the first column delivers the study's functional unit; and the background does not depend on the foreground; to be an

The augmented $\tilde{A}$ and $\tilde{B}$, makes up an \emph{LCA foreground study}, for which $\tilde{\mathbf{y}}$ is the canonical functional unit.  This formulation can be applied to the vast majority of LCA case studes, and very likely to all case studies prepared using commercial LCA software.  

It is currently common practice for background database maintainers to pre-compute the life cycle inventory results for their databases.  Using the notation $B_x = B \times (I - A)^{-1}$ to represent the pre-computed background LCI database, Eq.s~\ref{eqn:foreground} can be rewritten as:
\begin{equation}
 \tilde{A}_{flat} = \left[\begin{array}{cc} 
A_f &  0 \\
A_d & 0 
   \end{array}\right];\;\;\;  \tilde{B}_{flat} = \left[\begin{array}{cc} B_f & B_x \end{array}\right ]
\label{eqn:fg_bg}
\end{equation}
The proof of the equivalency of results derived from Eq.s~\ref{eqn:foreground} and \ref{eqn:fg_bg} is provided in the supplementary materials.  In this form, the main content of the LCIDB has been removed from the $\tilde{A}$ matrix, and the only information remaining in $\tilde{A}_{flat}$ pertains to the study foreground.  Furthermore, the final demand vector has been replaced with a canonical functional unit and no longer needs to be explicitly reported.

Using the ordered matrices, Eq.~\ref{eqn:leontief} can be written so that the computationally costly foreground and background inversions can be performed separately:
\begin{equation}
s = \mathbf{e}^T \times (B_f + B_x\times A_d) \times (I - A_f)^{-1} \times \tilde{\mathbf{y}}_f
\label{eqn:study}
\end{equation}
where $\tilde{\mathbf{y}}_f$ is a canonical functional unit vector having the same rank as $A_f$. Eq.~\ref{eqn:study} is derived in the supplementary materials.
Eq.~\ref{eqn:study} is visualized in Figure~\ref{fig:foreground}, with the background technology matrix $A$, having been subsumed into $B_x$, represented by an identity matrix $I$.  

The activity levels of the foreground nodes are represented by $\tilde{\mathbf{x}}$, which is the result of $\tilde{\mathbf{y}}_f$ selecting the first column of $(I-A_f)^{-1}$.
%% An important intermediate result is the vector of activity levels for the foreground nodes $\tilde{\mathbf{x}}$, also visualized in Figure~\ref{fig:foreground}:
%% \begin{equation}
%% \tilde{\mathbf{x}} = (I - A_f)^{-1} \times \tilde{\mathbf{y}}
%% \label{eqn:inv}
%% \end{equation}
This vector can be used to describe an \emph{aggregated foreground}:
\begin{equation}
\begin{array}{rl}
    \tilde{\mathbf{b}}_f & = B_f \times \tilde{\mathbf{x}} \\
    \tilde{\mathbf{a}}_d & = A_d \times \tilde{\mathbf{x}} \\
    \tilde{\mathbf{b}}_x & = B_x \times \tilde{\mathbf{a}}_d
\end{array}
\label{eqn:agg}
\end{equation}
The study result can also be computed from the aggregated foreground:
\begin{equation}
\begin{array}{rl}
s & = \mathbf{e}^T \times( B_f + B_x \times A_d) \times \tilde{\mathbf{x}} \\
  & = \mathbf{e}^T \times( \tilde{\mathbf{b}}_f + B_x \times\tilde{\mathbf{a}}_d) \\
  & = \mathbf{e}^T \times( \tilde{\mathbf{b}}_f + \tilde{\mathbf{b}}_x) \\
\end{array}
\end{equation}
