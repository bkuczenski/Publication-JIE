\subsection{Foreground Study}

In practice, most LCA studies contain information that is not present in a background database, and LCA software systems generally allow users to augment the standard LCIDBs with their own modeling information, known as the study foreground.  To meet our disclosure objectives, it is important to be able to distinguish foreground content from background content in the computation.  Background databases are prepared independently of any study and do not typically vary across the studies in which they are used.  However, the results of a study with foreground content cannot generally be computed without knowledge of the background.  %Put another way, the foreground results depend on the background but the background results do not depend on the foreground.  This condition allows the processes in an LCIDB to be partially ordered to group foreground processes together \citep{Kuczenski_JLCA_2015}.

To understand how a foreground study is constructed on top of a background database, imagine a simple product system that can be fully described by only making reference to processes that exist in a background database.  For instance, the use of a computer may be modeled using reference processes for producing and disposing the computer, and a reference process for electricity consumption.  Assuming those reference processes are available in a background database, the product system can be represented entirely by specifying a final demand vector $\mathbf{y}$ whose nonzero values correspond to computer production, computer disposal, and electricity.  The LCIA results can be computed using Eq.~\ref{eqn:leontief}.

The same result can be obtained by relocating the $\mathbf{y}$ vector into an augmented $A$ matrix, augmenting the $B$ matrix with a vector of zeros, and replacing the final demand with a vector $\tilde{\mathbf{y}}$ in which the first entry is 1 and all successive entries are 0:
\begin{equation}
  \begin{array}{cc}
s =& \mathbf{e}^T\times B\times\left(I-A\right)^{-1}\times\mathbf{y} \\
 =& \mathbf{e}^{T}\times\tilde{B}\times\left(I-\tilde{A}\right)^{-1}\times\tilde{\mathbf{y}}
  \end{array}
  \label{eqn:canonical}
\end{equation}
where
$\tilde{A} = \left[\begin{smallmatrix} 0 & \mathbf{0}^T \\ \mathbf{y} & A  \end{smallmatrix}\right]$,
$\tilde{B} = [ \mathbf{0},\, B ]$, and $\tilde{\mathbf{y}} =  [ 1,\, 0,\, 0 ,\,\ldots,\, 0]^{T}$.
It can easily be shown that Eq.~\ref{eqn:canonical} holds for any $\mathbf{y}$  (A proof is provided in the electronic supporting information).  This is intuitively true because the $\tilde{\mathbf{y}}$ selects the first column of the augmented $\tilde{A}$ matrix as the ``final demand,'' but the ``direct requirements'' of that column are precisely the final demand from the prior, un-augmented $A$ matrix.  

This same workflow can be repeated to build a more elaborate foreground.  For instance, a model of private vehicle travel may make use of reference processes for vehicle production and decommissioning, road maintenance, fuel retail, and a model of direct emissions from fuel combustion.  This model could occupy an additional column of $\tilde{A}$, with the coefficients for fuel combustion emissions being contained in the corresponding column of $\tilde{B}$.  A process for brewing coffee could be created in a third column that requires coffee beans, water, and electricity. Finally, a fourth column could be added that invokes the first three columns in proportion to describe the activity of traveling to a coffeeshop to work\footnote{No prizes for guessing where this passage was written.}.

In general, a product system model may be constructed through the successive augmentation of a background database with foreground content.  As long as the background database is not altered to depend on the newly created foreground, such a study can be written in block triangular form \citep{Kuczenski_JLCA_2015}:
\begin{equation}
\tilde{A} = \left[\begin{array}{cc}
A_f & 0 \\
A_d &  A
  \end{array}
\right];\;\;\;
  \tilde{B} = \left[\begin{array}{cc} B_f & B   \end{array}\right]
\label{eqn:foreground}
\end{equation}
Here, the submatrix $A_f$ represents the foreground; $A$ represents the background; the rectangular matrix $A_d$ represents the dependency of the foreground on the background; and the top right submatrix is zero.  The ordered $\tilde{B}$ matrix is similarly partitioned into $B_f$, which includes foreground emissions, and $B$, which includes background emissions. %The constitutive characteristic of this formulation is that the submatrix in the top right corner of $\tilde{A}$ is zero.
The augmented $\tilde{A}$ and $\tilde{B}$ together make up an \emph{LCA foreground study}.  This formulation can be applied to the vast majority of LCA case studies, and very likely to all case studies prepared using commercial LCA software.\footnote{It does not apply to cases in which the background database must be altered to adjust for double counting, such as in so-called ``integrated hybrid'' models in which all quadrants contain non-zero cells \citep{Suh2004}.  These models are more complex than the models typically built by LCA practitioners, and require more sophisticated computation.  In conventional LCA practice it is inadvisable to alter background databases to adapt them to study conditions.  Instead, background processes can be replicated in the foreground, where they can be adapted as needed (cf. \cite{Bourgault_JLCA_2013}), without affecting the integrity of the background database.}

Without loss of generality, it is possible to construct the study such that the \textit{first} column of $\tilde{A}$ delivers the functional unit of the study.  We thereby designate the vector $\tilde{\mathbf{y}} =  [ 1,\, 0,\, 0 ,\,\ldots,\, 0]^{T}$ as the \textit{canonical functional unit} of a foreground study, which allows us to introduce convenient notation later on.  

\begin{figure}
  \begin{center}
    \input{figures/fig1-foreground.fig}
    \caption{Matrix structure of an LCA foreground study.  The background technology matrix $X$ has been replaced with a background LCI database $B_x$.  The foreground input matrices and aggregation results are highlighted.}
    \label{fig:foreground}
  \end{center}
\end{figure}

It is currently common practice for background database maintainers to pre-compute the life cycle inventory results for their databases.  In cases where the background database is used without modification, including the contents of $A$ in $\tilde{A}$ becomes unnecessary.
Using the notation $B_x = B \times (I - A)^{-1}$ to represent the pre-computed background LCI database, %% $\tilde{A}$ can be ``flattened'' so that it no longer includes inventories for background processes.
%% Eq.s~\ref{eqn:foreground} can be recast as:
%% \begin{equation}
%%  \tilde{A}_{flat} = \left[\begin{array}{cc} 
%% A_f &  0 \\
%% A_d & 0 
%%    \end{array}\right];\;\;\;  \tilde{B}_{flat} = \left[\begin{array}{cc} B_f & B_x \end{array}\right ]
%% \label{eqn:fg_bg}
%% \end{equation}
%% The proof of the equivalency of results derived from Eq.s~\ref{eqn:foreground} and \ref{eqn:fg_bg} is provided in the supplementary materials.  In this form, the only information remaining in $\tilde{A}_{flat}$ pertains to the study foreground.  
%
%% Using the ordered matrices, 
Eq.~\ref{eqn:leontief} can be written so that the foreground matrix inversion is  performed separately from the computationally costly background computation:
\begin{equation}
s = \mathbf{e}^T \times (B_f + B_x\times A_d) \times (I - A_f)^{-1} \times \tilde{\mathbf{y}}_f
\label{eqn:study}
\end{equation}
where $\tilde{\mathbf{y}}_f$ is a canonical functional unit vector whose length is the same as the rank of $A_f$. Eq.~\ref{eqn:study} is derived in the supporting information.
Eq.~\ref{eqn:study} is visualized in Figure~\ref{fig:foreground}, with the background technology matrix $A$ subsumed into $B_x$ and replaced by an identity matrix $I$.  

The activity levels of the foreground nodes are represented by $\tilde{\mathbf{x}}$, which is the result of $\tilde{\mathbf{y}}_f$ selecting the first column of $(I-A_f)^{-1}$.
This vector can be used to describe an \emph{aggregated foreground} in coefficient form, which delivers the same results as the foreground study given in Eq.~\ref{eqn:foreground}, but without disclosing any details of the construction of the foreground:
\begin{equation}
\begin{array}{rl}
    \tilde{\mathbf{b}}_f & = B_f \times \tilde{\mathbf{x}} \\
    \tilde{\mathbf{a}}_d & = A_d \times \tilde{\mathbf{x}} \\
    \tilde{\mathbf{b}}_x & = B_x \times \tilde{\mathbf{a}}_d
\end{array}
\label{eqn:agg}
\end{equation}
Here $\tilde{\mathbf{a}}_d$ reports the aggregated foreground's dependency on background processes in background flows per canonical functional unit;  $\tilde{\mathbf{b}}_f$ reports foreground emissions per functional unit; and $\tilde{\mathbf{b}}_x$ reports the aggregated background life cycle emissions per functional unit.

The aggregated result can also be computed from the aggregated foreground:
\begin{equation}
\begin{array}{rl}
s & = \mathbf{e}^T \times( B_f + B_x \times A_d) \times \tilde{\mathbf{x}} \\
  & = \mathbf{e}^T \times( \tilde{\mathbf{b}}_f + B_x \times\tilde{\mathbf{a}}_d) \\
  & = \mathbf{e}^T \times( \tilde{\mathbf{b}}_f + \tilde{\mathbf{b}}_x) \\
\end{array}
\end{equation}
