\section{On the Equivalence of Process and Input-Output LCA}

There is a level of confusion about the differing implications of process-based LCA inventory analysis as described in Heijungs and Suh (\citeyear{Heijungs2002}), and input-output analysis using a Leontief matrix.  In fact, these two techniques were shown to be equivalent in Suh et al, (\citeyear{Suh_JIE_2010}), whose abstract declares: ``the article shows that the partitioning method in LCA is equivalent to the industry-technology model in input-output economics, and system expansion in LCA is equivalent to the by-product-technology model in input-output output economics.''  The article does indeed show convincingly that ``the two data types can be hybridized without the loss of methodological consistency'' (p.345).

   Both the technology matrix $A$ in LCA and the Leontief matrix $(I-A)$ in input output analysis are ways of encoding the relationships between the products that some processes ``make'' and other processes ``use.''  In either the classical or input-output formulation, both matrices must be put into symmetric form to be invertible, and this means describing each column as having a single, reference output.  This is done either through disaggregation (i.e. dividing a process by its various outputs), aggregation (i.e. combining products to form a market or industry), or some combination of these.  While the task of transforming data is highly case-dependent, it is not constrained by the choice of mathematical formalism.  
   
   Suh et al, after a rich investigation of the literature, a thorough mathematical development, and a worked example, conclude that ``even very large-scale LCI problems, such as those in commercial LCI databases, can be computed with a consistent mathematical framework$\ldots$'' that being the make-and-use framework (p. 349).  In effect, the different mathematical approaches can be made equivalent.  The equivalency can be made very plain by simply reinterpreting the input-output expressions \textit{as} technology matrices.  In other words, (in Suh et al) the expressions $(I-A_I)$ (eq. 6), $(I-A_C)$ (eq. 9), $(I-A_B)$ (eq. 12), and $(V'-U)$ (eq. 17) can simply be interpreted as \textit{different ways to formulate} the process-based technology matrix.  There is no inconsistency between this interpretation and the classical formulation.  

   Once single-output processes are obtained, one substantive difference between the various approaches is column scaling: whether the entries describe real outputs, unit outputs, or unit activity levels.  The database creator must ensure that a consistent scaling is applied to both the technology matrix and the emission matrix $B$.  The convention in LCA is for a column to represent a unit activity level.  To use the input-output coefficient approach, as in the present article, the columns must be scaled to a \textit{unit of the reference product}.  This reference output is then omitted from the $A$ matrix, and is instead supplied by the $I$ in the expression $(I-A)$.  This formulation is completely consistent with any other matrix-based formulation of LCA, including classical process LCA as traditionally practiced.  In particular, the approaches described here apply equally well to LCA models using both process-based databases such as Ecoinvent, input-output databases such as Exiobase, and background databases such as the GaBi databases.

   Interested readers should note that the canonical reference for LCA, ISO 14044, has nothing to say about the mathematical techniques used to solve LCA problems, and the only preference for a technology matrix over an input-output matrix is borne of tradition or custom rather than theory.  SimaPro, the leading worldwide software system for scientific LCA, converted to using an iterative Leontief inverse for Version 8, as disclosed in a whitepaper published in 2014\footnote{\url{https://www.pre-sustainability.com/news/new-calculation-engine-simapro-8}}, realizing significant speed and memory usage improvements.  OpenLCA, the open-source alternative, also reported a dramatic reduction in memory usage and computation time around the release of version 1.4 in 2014, coincident with a revised algorithm ``optimized for huge databases.''\footnote{\url{http://www.openlca.org/openlca-beta-1-4-released-optimised-for-huge-databases/}} This is highly suggestive of a switch to an iterative approach.  GaBi, the more industrially oriented software, does not use a matrix formulation at all, but has always computed activity levels using iteration at the plan level.
