\section{Examples}

LCI databases often contain product systems that can be modeled as foreground studies because they describe products that are not required by the background.  In this section, two product systems selected from LCI databases are used to illustrate the concept of structured publication.  The systems selected because are complex enough to illustrate the premise but simple enough to review easily.

Each system is illustrated as a table that shows the foreground model, cutoff flows, background dependencies and foreground emissions included in the system.  Aggregation results $\tilde{\mathbf{x}}$, $\tilde{\mathbf{a}}_d$, and $\tilde{\mathbf{b}}_f$ are also reported.  The table omits numeric data from most of the table for clarity. Instead, a black square indicates the presence of a nonzero value.  %The systems are also reproduced as structured publications in the supplementary materials, formatted as Excel spreadsheets.

\subsection{US LCI: Secondary Aluminum}

\begin{figure}
Aluminum, secondary, ingot, from automotive scrap, at plant [RNA]

{\scriptsize\sffamily
\begin{tabularx}{\textwidth}{|>{\hangindent=2ex}X|c@{~}c@{~}c@{~}c@{~}|c|}
\hline
\everypar{\hangindent1em \hangafter1}
(node) Foreground flow \rule[-3pt]{0pt}{12pt} & 0 & 1 & 2 & 3 & \\ 
\hline
(0) Aluminum, secondary, ingot, from automotive scrap, at plant (kg) [RNA] & \refbox  &  &  &  & \\ 
(1) Aluminum recovery, transport, to plant (kg) [RNA] & 1.03 & \refbox  &  &  & \\ 
(2) Quicklime, at plant (kg) [RNA] & 2.35e-05 &  & \refbox  &  & \\ 
(3) Limestone, at mine (kg) [RNA] &  &  & 1.87 & \refbox  & \\ 
\hline
Foreground Node Weights $\tilde{x}$ &    1 & 1.03 & 2.35e-05 & 4.39e-05& \\ 
\hline
Input: CUTOFF Disposal, solid waste, unspecified, to sanitary landfill [CUTOFF Flows] & \dependency &  & \dependency &  & 0.0842\\ 
Input: CUTOFF Filter media, at plant [CUTOFF Flows] & \dependency &  &  &  & 4.79e-05\\ 
Input: CUTOFF Lube oil, at plant [CUTOFF Flows] & \dependency &  &  &  & 8.75e-07\\ 
Input: CUTOFF Treatment gases, unspecified, at plant [CUTOFF Flows] & \dependency &  &  &  & 0.003\\ 
Input: CUTOFF Alloying additives, at plant [CUTOFF Flows] & \dependency &  &  &  & 0.021\\ 
Input: CUTOFF Chemicals, unspecified, used for wastewater treatment [CUTOFF Flows] & \dependency &  &  &  & 0.008\\ 
Input: CUTOFF Grain refiners, at plant [CUTOFF Flows] & \dependency &  &  &  & 0.0003\\ 
Input: CUTOFF Treatment salts, unspecified, at plant [CUTOFF Flows] & \dependency &  &  &  & 0.009\\ 
Input: CUTOFF Packaging, unspecified, at plant [CUTOFF Flows] & \dependency &  &  &  & 1.67e-05\\ 
Input: CUTOFF Aluminum scrap, automotive [CUTOFF Flows] & \dependency &  &  &  &  1.03\\ 
\hline
\hline
Background Dependencies \rule[-3pt]{0pt}{12pt} & 0 & 1 & 2 & 3 & $\tilde{a_d}$\\ 
\hline
Diesel, combusted in industrial boiler (l) [RNA] &  &  & \dependency & \dependency & 4.79e-08\\ 
Transport, train, diesel powered (t*km) [RNA] &  & \dependency & \dependency &  & 0.0415\\ 
Electricity, at grid, US, 2000 (kWh) [RNA] & \dependency &  & \dependency & \dependency & 0.668\\ 
Transport, combination truck, diesel powered (t*km) [RNA] &  & \dependency & \dependency &  & 0.374\\ 
Natural gas, combusted in industrial boiler (m3) [RNA] & \dependency &  & \dependency & \dependency & 0.223\\ 
Bituminous coal, combusted in industrial boiler (kg) [RNA] &  &  & \dependency & \dependency & 4.04e-06\\ 
Gasoline, combusted in equipment (l) [RNA] &  &  &  & \dependency & 2.25e-09\\ 
Liquefied petroleum gas, combusted in industrial boiler (l) [RNA] &  &  & \dependency &  & 7.57e-10\\ 
Transport, barge, average fuel mix (t*km) [RNA] &  &  & \dependency &  & 5.66e-07\\ 
\hline
Foreground Emissions \rule[-3pt]{0pt}{12pt} & 0 & 1 & 2 & 3 & $\tilde{b_f}$\\ 
\hline
Output: Lead [air, unspecified] & \dependency &  &  &  & 2.16e-07\\ 
Output: Suspended solids, unspecified [water, unspecified] & \dependency &  &  &  & 2.56e-05\\ 
Output: Particulates, unspecified [air, unspecified] & \dependency &  & \dependency & \dependency & 1.78e-07\\ 
Output: NMVOC, non-methane volatile organic compounds [air, unspecified] & \dependency &  &  &  & 4.7e-05\\ 
Output: BOD5, Biological Oxygen Demand [water, unspecified] & \dependency &  &  &  & 1.35e-09\\ 
Output: COD, Chemical Oxygen Demand [water, unspecified] & \dependency &  &  &  & 8.3e-07\\ 
Output: Sulfur dioxide  [air, unspecified] &  &  & \dependency &  & 3.52e-09\\ 
Output: Dissolved solids [water, unspecified] & \dependency &  &  &  & 2.38e-08\\ 
Output: carbon dioxide [air, unspecified] &  &  & \dependency &  & 1.8e-05\\ 
Input: Limestone [resource, ground-] &  &  &  & \dependency & 4.39e-05\\ 
Output: Heavy metals, unspecified [water, unspecified] & \dependency &  &  &  & 2.25e-05\\ 
Output: Organic substances, unspecified [water, unspecified] & \dependency &  &  &  & 1.29e-06\\ 
Output: Acids, unspecified [air, unspecified] & \dependency &  &  &  & 4.31e-05\\ 
\hline
\end{tabularx}
}

  \caption{A structured product system model for secondary aluminum, drawn from US LCI. Exchange values are replaced with black squares for clarity.}
  \label{table:aluminum}
\end{figure}


The US LCI database contains a small background system of 39 processes, as well as 395 foreground product systems that range in size from one to 83 foreground flows.
%The largest, ``paper, freesheet, coated, at mill'' makes reference to 34 of the 39 background processes.

The example product system  reports the production of secondary aluminum from automotive scrap (Figure~\ref{table:aluminum}).   The aluminum production process has direct requirements for two other foreground systems, including transportation services and quicklime production.  The quicklime production itself depends on limestone extraction, which is also part of the foreground.  The aluminum production requires two background systems, electricity and natural gas combustion, and reports 10 cutoff flows.  The main input to the process, ``Aluminum scrap, automotive,'' appears as a cutoff flow.  Overall, the four foreground processes require the background for only combustion (five fuels), transport (three modes) and grid electricity.  The same natural gas combustion and electricity models were used in all three non-transport foreground nodes.

The system also reports several direct emissions, mainly from the aluminum process.  Only one resource consumption (input flow) is reported -- the extraction of limestone. The direct emission modeling is limited in scope and includes a number of ``unspecified'' flows that may not be well characterized in many impact methods.

%The structured publication of the secondary aluminum model includes a total of 47 data points about 36 entities.  The $A_f$, $A_d$, and $B_f$ matrices are reported in sparse format, along with the aggregation vectors. In the publication, LCIA scores are reported for several impact indicators drawn from the Ecoinvent reference LCIA implementation.\footnote{The Ecoinvent LCIA factors are not published, but are available from the Ecoinvent Centre. The name of the file used in the current study is \texttt{LCIA implementation v3.1 2014\_08\_13.xlsx}.}    The $E$ matrix is also partially reproduced: entries are only reported if they match flows that appear in the foreground.

\subsection{Ecoinvent: Organic Potatoes}

\begin{figure}
  market for potato, organic [GLO]


{\scriptsize\sffamily
\begin{tabularx}{\textwidth}{|>{\hangindent=3ex}X|c@{~}c@{~}c@{~}c@{~}c@{~}c@{~}c@{~}c@{~}c@{~}|c|}
\hline
(node) Foreground flows -- $A_f$ \rule[-3pt]{0pt}{12pt} & 0 & 1 & 2 & 3 & 4 & 5 & 6 & 7 & 8 & \\ 
\hline
(0) potato, organic [GLO] (kg) & \refbox  &  &  &  &  &  &  &  &  & \\ 
(1) potato, organic [CH] (kg) & 0.024 & \refbox  &  &  &  &  &  &  &  & \\ 
(2) potato, organic [RoW] (kg) & 0.976 &  & \refbox  &  &  &  &  &  &  & \\ 
(3) potato seed, organic, at farm [CH] (kg) &  &  &  & \refbox  & 0.024 &  &  &  &  & \\ 
(4) potato seed, organic, at farm [GLO] (kg) &  &  &  &  & \refbox  &    1 &    1 &  &  & \\ 
(5) potato seed, organic, for setting [RoW] (kg) &  &  &  &  &  & \refbox  &  &  & 0.976 & \\ 
(6) potato seed, organic, for setting [CH] (kg) &  &  &  &  &  &  & \refbox  &  & 0.024 & \\ 
(7) potato seed, organic, at farm [RoW] (kg) &  &  &  &  & 0.976 &  &  & \refbox  &  & \\ 
(8) potato seed, organic, for setting [GLO] (kg) &  & 0.11 & 0.11 & 0.16 &  &  &  & 0.16 & \refbox  & \\ 
\hline
Foreground Node Weights $\tilde{x}$ &    1 & 0.024 & 0.976 & 0.00315 & 0.131 & 0.128 & 0.00315 & 0.128 & 0.131& \\ 
\hline
\hline
Background Dependencies -- $A_d$ \rule[-3pt]{0pt}{12pt} & 0 & 1 & 2 & 3 & 4 & 5 & 6 & 7 & 8 & $\tilde{a_d}$\\ 
\hline
building, multi-storey [GLO] (m3) &  &  &  &  &  & \dependency & \dependency &  &  & 5.24e-06\\ 
electricity, low voltage [GLO] (kWh) &  &  &  &  &  & \dependency &  &  &  & 0.00933\\ 
potato haulm cutting [GLO] (ha) &  & \dependency & \dependency & \dependency &  &  &  & \dependency &  & 5.04e-05\\ 
tillage, hoeing and earthing-up, potatoes [GLO] (ha) &  & \dependency & \dependency & \dependency &  &  &  & \dependency &  & 0.000101\\ 
transport, tractor and trailer, agricultural [GLO] (metric ton*km) &  & \dependency & \dependency & \dependency &  &  &  & \dependency &  & 0.00113\\ 
transport, freight, light commercial vehicle [GLO] (metric ton*km) & \dependency &  &  &  & \dependency &  &  &  &  & 0.0258\\ 
tillage, ploughing [GLO] (ha) &  & \dependency & \dependency & \dependency &  &  &  & \dependency &  & 5.04e-05\\ 
green manure, organic, until March [GLO] (ha) &  & \dependency & \dependency & \dependency &  &  &  & \dependency &  & 5.04e-05\\ 
transport, freight, lorry, unspecified [GLO] (metric ton*km) & \dependency &  &  &  & \dependency &  &  &  &  & 0.404\\ 
transport, freight train [GLO] (metric ton*km) & \dependency &  &  &  & \dependency &  &  &  &  & 0.138\\ 
transport, freight, inland waterways, barge [GLO] (metric ton*km) & \dependency &  &  &  & \dependency &  &  &  &  & 0.0857\\ 
copper oxide [GLO] (kg) &  & \dependency & \dependency & \dependency &  &  &  & \dependency &  & 0.000107\\ 
tillage, harrowing, by spring tine harrow [GLO] (ha) &  & \dependency & \dependency & \dependency &  &  &  & \dependency &  & 5.04e-05\\ 
potato planting [GLO] (ha) &  & \dependency & \dependency & \dependency &  &  &  & \dependency &  & 5.04e-05\\ 
solid manure loading and spreading, by hydraulic loader and spreader [GLO] (kg) &  & \dependency & \dependency & \dependency &  &  &  & \dependency &  & 0.717\\ 
potato grading [GLO] (kg) &  & \dependency & \dependency & \dependency &  &  &  & \dependency &  &  1.13\\ 
tillage, harrowing, by rotary harrow [GLO] (ha) &  & \dependency & \dependency & \dependency &  &  &  & \dependency &  & 5.04e-05\\ 
harvesting, by complete harvester, potatoes [GLO] (ha) &  & \dependency & \dependency & \dependency &  &  &  & \dependency &  & 5.04e-05\\ 
transport, freight, sea, transoceanic ship [GLO] (metric ton*km) & \dependency &  &  &  & \dependency &  &  &  &  & 0.525\\ 
liquid manure spreading, by vacuum tanker [GLO] (m3) &  & \dependency & \dependency & \dependency &  &  &  & \dependency &  & 0.000565\\ 
electricity, low voltage [CH] (kWh) &  &  &  &  &  &  & \dependency &  &  & 0.00023\\ 
irrigation [GLO] (m3) &  &  & \dependency &  &  &  &  & \dependency &  & 0.0173\\ 
tillage, currying, by weeder [GLO] (ha) &  & \dependency & \dependency & \dependency &  &  &  & \dependency &  & 0.000101\\ 
application of plant protection product, by field sprayer [GLO] (ha) &  & \dependency & \dependency & \dependency &  &  &  & \dependency &  & 0.000222\\ 
irrigation [CH] (m3) &  & \dependency &  & \dependency &  &  &  &  &  & 0.000426\\ 
\hline
Foreground Emissions -- $B_f$ \rule[-3pt]{0pt}{12pt} & 0 & 1 & 2 & 3 & 4 & 5 & 6 & 7 & 8 & $\tilde{b_f}$\\ 
\hline
Input: Occupation, construction site [natural resource, land] (m2*year) &  &  &  &  &  & \dependency & \dependency &  &  & 2.1e-06\\ 
Input: Transformation, from unspecified [natural resource, land] (m2) &  &  &  &  &  & \dependency & \dependency &  &  & 1.05e-06\\ 
Input: Transformation, to industrial area [natural resource, land] (m2) &  &  &  &  &  & \dependency & \dependency &  &  & 1.05e-06\\ 
Input: Energy, gross calorific value, in biomass [natural resource, biotic] (MJ) &  & \dependency & \dependency & \dependency &  &  &  & \dependency &  &  3.87\\ 
Output: Phosphate [water, ground-] (kg) &  & \dependency & \dependency & \dependency &  &  &  & \dependency &  & 6.25e-06\\ 
Output: Cadmium, ion [water, ground-] (kg) &  & \dependency & \dependency & \dependency &  &  &  & \dependency &  & 1.14e-09\\ 
Output: Zinc, ion [water, ground-] (kg) &  & \dependency & \dependency & \dependency &  &  &  & \dependency &  & 1.35e-06\\ 
Output: Nitrate [water, ground-] (kg) &  & \dependency & \dependency & \dependency &  &  &  & \dependency &  & 0.00696\\ 
$\ldots$ (31 rows omitted) &  &  &  &  &  &  &  &  & \\ 
\hline
\end{tabularx}
}

  \caption{A structured product system model for organic potato production, drawn from Ecoinvent v3.2 (APOS). Exchange values are replaced with black squares for clarity.}
  \label{table:potato}
\end{figure}


Ecoinvent version 3 is provided in three different system models that reflect different linking strategies.  The example is drawn from the ``Allocation at the point of substitution'' or APOS model of Ecoinvent 3.2, which includes 11,420 processes that produce 12,966 product flows.  Of these, 10,282 are background flows and the rest are foreground flows.

The example system reports production of organic potatoes supplied to the global market (Figure \ref{table:potato}).  The foreground includes nine nodes, of which six (nodes 3 through 8) contain cyclic dependencies associated with the production of potato seeds.  The Ecoinvent database's use of ``markets'' as mixer processes is evident in the table: node 0 is a mixer process that combines Swiss (``CH'') production (node 1) weighted at 2.4\%  with rest-of-World (``RoW'') production (node 2) weighted at 97.6\%.  That same market split (2.4 / 97.6) can also be seen in nodes 4 and 8.

Nodes 5 and 6, which make seeds ready ``for setting,'' are mixed by node 8. Although nodes 5 and 6 are geographically distinguished (CH vs RoW), both can be seen to consume potato seed from the global market.

Looking at the dependency and emission lists, the different ``signatures'' of different kinds of processes can be seen: nodes 1, 2, 3, and 7 are clearly agricultural processes that require irrigation, tillage, manure and so on.  Nodes 0 and 4 are visible as market processes, their only requirements being transport processes.  Nodes 5 and 6 each consume electricity and require use of a ``multi-storey building.''  Taken together, the CH-locality processes appear to use CH irrigation and electricity supply but are otherwise similar to their RoW counterparts.  The product model includes no cutoff flows.

%The structured publication of the organic potato model reports the $A_f$, $A_d$, and $B_f$ matrices in full rather than sparse format to enable a more visually immediate review.  In addition, the $E$ matrix is reported in full, for all emissions found in the model, rather than only the emissions found in the foreground.  This enables the publication to be used to visually evaluate different implementations of the same impact assessment method in the context of the inventory model (Ecoinvent vs ILCD implementations of ReCiPe Midpoint marine eutrophication; see supplementary materials for details).
