\begin{abstract}
  Many of the challenges facing knowledge synthesis from life cycle assessment (LCA) studies stem from the inability of study authors and readers to formally agree on the structure and content of the product system models used to perform LCA computations.
  This article presents a framework for formally disclosing the foreground of an LCA study in a way that permits the computations to be inspected, verified, and reproduced by a reader, provided that the reader has access to the same life cycle inventory and impact characterization resources as the author.
  The framework can also be used to partition a study into public and private portions, allowing both portions to be critically reviewed but omitting the private information from the disclosure.
  A disclosure is made up of six components, including three lists of entities in the model and three sparse matrices describing their interconnections.  The entity lists make reference to previously-published resources, including background inventory databases and characterized elementary flows, and the disclosure framework requires both author and reader to agree on the meaning of each of these references.
  The framework contributes to ongoing efforts within and beyond industrial ecology to improve the reproducibility and verifiability of scholarly works, and if implemented, plots a course toward distributed, platform-independent computation and validation of LCA results.
\end{abstract}


\endinput

The author's 


disclosing the portions of an LCA study that embody the work of the author
