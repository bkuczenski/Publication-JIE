\subsection{Mathematical Formulation of LCA}

In the most general terms, the LCA problem concerns the interaction of a collection of activities and commodities described by a supply-and-use inventory, in which $U$ is a table of the use of each commodity by each activity (inputs) and $V$ is a table of the supply of each commodity by each activity (outputs).  Such an account can always be transformed into a \textit{symmetric} table of inputs and outputs on a commodity-by-commodity or industry-by-industry basis \citep{Eurostat_2008}. Doing so requires the application of various assumptions and transformations to the data, the details of which are not in scope of this paper, but which have recently been reviewed in depth \citep{Suh_JIE_2010, Majeau_Bettez_2014}.  

A symmetric input output table can be written as a table of coefficients, in which each column of the matrix provides a ``recipe'' for creating the corresponding commodity by consuming other commodities (and, if applicable, producing other co-products).  The table of coefficients, commonly known as a \textit{direct requirements matrix}, can be used to compute the output from all activities that is necessary to produce a given commodity.  That vector of outputs can then be combined with a second table of coefficients, known as an \textit{emission matrix}, that reports exchanges with the environment associated with the production of each commodity, to determine the cumulative environmental emissions associated with the production of the commodity over its life cycle.  Those emissions are individually \textit{characterized} with respect to an environmental impact category of interest, and the characterized emissions summed to report the \textit{impact score} or category indicator, which is the result of the study.  The formulation may be summarized:
\begin{equation}
s = \mathbf{e}^T \cdot B \cdot \left(I - A\right)^{-1} \cdot \mathbf{y}
\label{eqn:leontief}
\end{equation}
Here $A$ represents the direct requirements; $B$ is the emission matrix;  $\mathbf{e}$ is a column vector of characterization factors for the environmental emissions; $\mathbf{y}$ is the externally specified final demand; and $s$ is the numerical impact score or category indicator.  For simplicity, we will consider the computation of a single impact result.  However, it is straightforward to imagine the more typical case, in which $E$ is a matrix of characterization vectors, and $\mathbf{s}$ is a vector of results.  Eq.~\ref{eqn:leontief} is visualized in Figure~\ref{fig:lca}.
\begin{figure}
  \begin{center}
    \input{figures/fig1-matrices.fig}
    \caption{Traditional matrix structure of an LCA computation.}
    \label{fig:lca}
  \end{center}
\end{figure}


The term $(I-A)$ is equivalent to a normalized \textit{technology matrix}, and in certain cases can be computed from the difference between the supply and use tables, $(V'-U)$, with equivalent results \citep{Suh_JIE_2010}. 
The coefficient matrices $A$ (or $I-A$) and $B$ together are called a \textit{life cycle inventory database} (LCIDB), and the functional unit of a study is given by the final demand vector $\mathbf{y}$, which is the product or service whose delivery is to be assessed.\footnote{%
Differences in convention have led to both the direct requirements matrix $A$, and the technology matrix, analogous to $(I-A)$, being denoted by the same symbol in different contexts.  If the data contained in the LCIDB is sufficiently rich, either approach may be implemented with equivalent computational results.  The direct requirement formulation has the advantage of permitting iterative solution, which can dramatically reduce the computational resources required \citep{Peters_JLCA_2007}.}  Intermediate results of this computation include:
\begin{itemize}
\item $\mathbf{x} = \left(I-A\right)^{-1}\cdot\mathbf{y}$, the activity level vector;
\item $\mathbf{b} = B\cdot\mathbf{x}$, the life cycle inventory;
\item $s = \mathbf{e}^T\cdot\mathbf{b}$, the life cycle impact category indicator, the study result.
\end{itemize}

This formulation implemented in all major LCA software systems, some of which provide the capacity to export the $A$ and $B$ matrices for a given study model.  However, the contents of commercial LCIDBs are proprietary and subject to licensing restrictions, and moreover are not usually modified by study authors.  Thus exporting $A$ and $B$ does not meet our authority and primacy objectives.  Additionally, replicating the entire LCIDB is cumbersome because of the large size of the matrices.   As a consequence, the matrix formulation in Eq.~\ref{eqn:leontief} is unsuitable as a disclosure framework.
