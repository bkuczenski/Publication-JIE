\subsection{Disclosure of Study Design}

The foreground matrix $A_f$, dependency matrix $A_d$, and foreground emission matrix $B_f$ now contain all study-specific information.  Assuming that the appropriate background LCIDB $B_x$, and the vector of characterization factors $\mathbf{e}$, are available and held in common between the author and the reader, then a disclosure that precisely describes $A_f$, $A_d$ and $B_f$ is sufficient to reproduce the study result, as formulated in Eq.~\ref{eqn:study}.  These three submatrices together are referred to as the product system model, or PSM.

This set of three submatrices can be used to represent the product system model; thus, the problem of how to disclose an LCA study design is reduced to the problem of how to communicate the contents of these submatrices accurately.  Recognizing that all three matrices are likely to be sparse (that is, that most of the entries are zero), the most efficient disclosure would be to only report non-zero entries.  A sparse matrix can be represented by a list of 3-tuples indicating (row, column, value) for each non-zero entry.  Thus, a formal representation of the PSM can be stated in six parts:
\begin{enumerate}[label=\alph*.]
\item\label{itm:fg} An ordered list of foreground nodes (rows/columns of $A_f$);
\item\label{itm:bg} An ordered list of background flows, making reference to a particular background database (rows of $A_d$, mapping to columns of $B_x$);
\item\label{itm:em} An ordered list of exchanges with environment (rows of $B_f$, mapping to entries in $\mathbf{e}$);
\item A list of 3-tuples for nonzero elements of $A_f$, in which the row and column are indices into Item~\ref{itm:fg};
\item a list of 3-tuples for nonzero elements of $A_d$, in which the rows are indices into Item~\ref{itm:bg} and the columns are indices into Item~\ref{itm:fg};
\item a list of 3-tuples for nonzero elements of $B_f$, in which the rows are indices into Item~\ref{itm:em} and the columns are indices into Item~\ref{itm:fg}.
\end{enumerate}

For a disclosure that includes the computation of an LCIA indicator or indicators, a seventh part is required:
\begin{enumerate}[resume]
\item the identity of the LCIA indicator or indicators computed (identity of $\mathbf{e}$).
\end{enumerate}

A disclosure containing these six elements satisfies most of the requirements established at the top of this section (the exception is ensuring the privacy of confidential data, which is treated later in the paper):
\begin{itemize}
  \item The result is \textbf{computable} using Eq.~\ref{eqn:study};
  \item the computation is \textbf{complete} and \textbf{minimal} because all components are required and no superfluous components are included;
  \item The results are \textbf{reproducible} if the reader is able to accurately interpret the author's references to $B_x$ and $\mathbf{e}$;
  \item The disclosure includes only information for which the author can claim \textbf{authority};
  \item The disclosure includes only information for which the study is the \textbf{primary} source.
\end{itemize}
As mentioned, the proper interpretation of the disclosure requires that the reader accurately interprets the author's references to background databases and elementary flows.  If LCIA indicator results are included, the author must also unambiguously identify the method computed.  Implications of this requirement are discussed below.



For each row in $A_d$ containing a non-zero entry, the author must unambiguously identify the version of the database used, as well as the exact process and reference flow used from the background database; the dimension (reference unit) of each reference flow must be specified; and the sign of the numeric entry in $A_d$ must be consistent with the implementation of the process in the background database.  For the reader's part, the reader must be able to obtain either the column of $B_x$, or the scalar value that 

Similarly, for each row in $B_f$ containing a non-zero entry, the author must unambiguously identify the substance being exchanged with the environment, the compartment or context into which it is being exchanged, and 



\subsection{Foreground Aggregation}

Often the contents of the foreground

An important intermediate result is the vector of activity levels for the foreground nodes $\tilde{\mathbf{x}}$, also visualized in Figure~\ref{fig:foreground}:
\begin{equation}
\tilde{\mathbf{x}} = (I - A_f)^{-1} \times \tilde{\mathbf{y}}
\label{eqn:inv}
\end{equation}
This vector can be used to describe an \emph{aggregated foreground}, which generates the same results as the fully expanded foreground:
\begin{equation}
\begin{array}{rl}
    \tilde{\mathbf{b}}_f & = B_f \times \tilde{\mathbf{x}} \\
    \tilde{\mathbf{a}}_d & = A_d \times \tilde{\mathbf{x}} \\
    \tilde{\mathbf{b}}_x & = B_x \times \tilde{\mathbf{a}}_d
\end{array}
\label{eqn:agg}
\end{equation}

These vectors summarize the contents of the foreground without disclosing its detailed structure.  The aggregated dependency vector $\tilde{\mathbf{a}}_d$ has the same dimension as the background database, and the aggregated emission vector $\tilde{\mathbf{b}}_f $has the same dimension as the elementary flow matrix.  The dependency vector can be transformed into an elementary flow vector by multiplying by $B_x$.  The result of the study $s$ is the sum of foreground and background impact scores:

\begin{equation}
\begin{array}{rl}
   \tilde{\mathbf{b}} & = \tilde{\mathbf{b}}_f + \tilde{\mathbf{b}}_x \\
   s &= \mathbf{e}^T \times \tilde{\mathbf{b}} \\
   & = \mathbf{e}^T \times \tilde{\mathbf{b}}_f + \mathbf{e}^T \times \tilde{\mathbf{b}}_x \\
   &= \tilde{s}_f + \tilde{s}_x
\end{array}
\label{eqn:lci}
\end{equation}



\subsubsection{Model fragments and subsystem boundaries}


The nodes that make up the foreground of an LCA study describe an implicit system boundary

Most complex


\section{Preparing and Reviewing Aggregation Results}



