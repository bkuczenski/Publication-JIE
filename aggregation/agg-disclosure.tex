\subsection{Contents of a Disclosure}

The foreground matrix $A_f$, dependency matrix $A_d$, and foreground emission matrix $B_f$ now contain all study-specific information.  Assuming that the appropriate background LCIDB $B_x$, and the vector of characterization factors $\mathbf{e}$ (or their product $\mathbf{e}\times B_x$) are available and held in common between the author and the reader, then a disclosure that precisely describes $A_f$, $A_d$ and $B_f$ is sufficient to reproduce the study result, as formulated in Eq.~\ref{eqn:study}.  These three submatrices together can be considered a complete representation of the PSM.

Thus, the problem of how to disclose an LCA study design is reduced to the problem of how to communicate the contents of these submatrices accurately.  Recognizing that all three matrices are likely to be sparse (that is, that most of the entries are zero), the most efficient disclosure would be to only report non-zero entries.  A sparse matrix can be represented by a list of 3-tuples indicating (row, column, value) for each non-zero entry.  Thus, a formal disclosure of the PSM can be stated in six parts:
\begin{enumerate}[label={\em d-\roman*}., ref={\em d-\roman*}]
\item\label{itm:fg} An ordered list of foreground nodes (rows/columns of $A_f$);
\item\label{itm:bg} An ordered list of background flows, making reference to a particular background database (rows of $A_d$, mapping to columns of $B_x$);
\item\label{itm:em} An ordered list of exchanges with environment (rows of $B_f$, mapping to entries in $\mathbf{e}$);
\item\label{itm:af} A list of 3-tuples for nonzero elements of $A_f$, in which the row and column are indices into Item~\ref{itm:fg};
\item\label{itm:ad} a list of 3-tuples for nonzero elements of $A_d$, in which the rows are indices into Item~\ref{itm:bg} and the columns are indices into Item~\ref{itm:fg};
\item\label{itm:bf} a list of 3-tuples for nonzero elements of $B_f$, in which the rows are indices into Item~\ref{itm:em} and the columns are indices into Item~\ref{itm:fg}.
\end{enumerate}

A disclosure containing these six elements satisfies most of the requirements established at the top of this section (the exception is ensuring the privacy of confidential data, which is treated later in the article):
\begin{itemize}
  \item The result is \textbf{computable} using Eq.~\ref{eqn:study};
  \item the computation is \textbf{complete} and \textbf{minimal} because all components are required and no superfluous components are included;
  \item The results are \textbf{reproducible} if the reader is able to accurately interpret the author's references in \ref{itm:bg} and \ref{itm:em};
  \item The disclosure includes only information for which the author can claim \textbf{authority};
  \item The disclosure includes only information for which the study is the \textbf{primary} source.  
\end{itemize}
As mentioned, the proper interpretation of the disclosure requires that the reader accurately interprets the author's references to background databases and elementary flows, including mutually consistent implementation of the LCIA vector $\mathbf{e}$.  These aspects are out of scope for the purposes of this paper, and we will assume that a shared understanding between author and reader can be reached.  However, the the challenges of operationalizing this requirement are complex (see Discussion and Supplementary materials).  %Some implications on the information required for a successful disclosure are considered in the Discussion.

