\subsection{Forms of Aggregation}

Figure~\ref{fig:foreground} illustrates the LCA computation in various successive aggregation steps, from full matrices to vectors to a scalar value.  The level of aggregation provided in a disclosure will constrain what review objectives can be met by the reader of a study.  Different aggregation forms are illustrated in Figure~\ref{fig:aggregation}.

\subsubsection{Unit Process Inventory}

A unit process makes up a single shared column of $A_f$, $A_d$, and $B_f$, as depicted in Figure~\ref{fig:aggregation}a.  Documentation of unit processes is specified in ISO 14048, and several compliant formats exist, the most well-known being the ILCD and Ecospold XML formats.  ISO 14048 specifies a distinction between intermediate and elementary flows, but does not include the concept of a model foreground.  In a matrix representation, each unit process must have exactly one reference flow, which corresponds to its row and column in $A_f$.

\begin{figure}[t]
  \begin{center}
    \input{fig4-aggregation.fig}
    \caption{Matrix structures for different forms of foreground aggregation.}
    \label{fig:aggregation}
  \end{center}
\end{figure}


\subsubsection{Complete Foreground}

See Figure~\ref{fig:aggregation}b.  A foreground can be viewed as a linked set of unit processes.  The collection of processes themselves make up the foreground of the study, and so exchanges among them are recorded in $A_f$, and exchanges outside the foreground are represented in $A_d$ or $B_f$.  A complete foreground represents
the full disclosure of a model, and a reader can replicate and extend it independently.
%An excerpt from a larger study may also be published as a complete foreground. %
Any modeling technique used to treat a multi-functional process, such as an allocation or system expansion, can only be fully expressed as a foreground with at least as many columns as outputs.  For examples of multi-output processes represented as foreground models, please see the supporting information.  The example shown in Figure~\ref{fig:aggregation}b shows five nodes, including one cut-off flow

\subsubsection{Aggregated Foreground}

See Figure~\ref{fig:aggregation}c. Any collection of foreground nodes can also be expressed in aggregated form by performing the computations in Eq.~\ref{eqn:agg}.  A disclosure of an aggregated foreground resembles the description of a unit process.  As in the unit process case, the aggregated foreground has exactly one reference flow, the canonical functional unit $\tilde{\mathbf{y}}$.  The aggregated foreground still reports explicit links between the model and a background database and can thus be used to review data set selection.

An aggregated foreground with cut-off flows resembles a multi-output process.  After aggregation, foreground flows that correspond to cut-offs must still be reported in a disclosure, while foreground flows interior to the model can be omitted.  This is done with the introduction of a new vector, $\tilde{\mathbf{a}}_f$, which includes the reference flow plus any entries from $\tilde{\mathbf{x}}$ that correspond to cut-off flows.  

\subsubsection{Partial Background Aggregation}

See Figure~\ref{fig:aggregation}d.  Using this approach, the
 aggregated dependency vector is split into two parts that sum to the original:
\begin{equation}
 \tilde{\mathbf{a}}_d = \tilde{\mathbf{a}}_{d,priv} + \tilde{\mathbf{a}}_d'
\end{equation}
The disclosed dependencies $\tilde{\mathbf{a}}_d'$ are reported, and the private dependencies $\tilde{\mathbf{a}}_{d,priv}$ are replaced with an aggregated background inventory derived from the background database, $\tilde{\mathbf{b}}_{x,agg}$.  This approach can be used when a reader lacks access to the background inventory sets referred to in $\tilde{\mathbf{a}}_{d,priv}$, or when the author wishes not to disclose the private dependencies.  Foreground emissions $\tilde{\mathbf{b}}_f$ can be reported separately from the aggregation result, but often in current practice they are not distinguished from the background flows.  

\subsubsection{Full Background Aggregation and LCI}

See Figure~\ref{fig:aggregation}e and f. At this level of aggregation, the entire dependency vector is replaced with an aggregated life cycle inventory vector $\tilde{\mathbf{b}}_x$ derived from the background LCIDB.  The reader no longer requires access to any background database to perform the computation, but all dependency information is concealed.  As mentioned above, foreground emissions are often not distinguished from the aggregation result; if foreground and background emissions are added together, the disclosure is as shown in Figure~\ref{fig:aggregation}f.  The life cycle inventory $\tilde{\mathbf{b}}$ provides the most aggregated form of the study that can still be independently validated with an external characterization vector $\mathbf{c}$.  

