\subsection{Stable Semantic References to Study Elements}

In practice, while it is easy to reproduce a set of sparse matrices, there is considerable potential for ambiguity in identifying the background datasets and emissions to which those matrix entries correspond.
In order to achieve the transparency and reproducibility promised by this disclosure framework, it must be possible for study authors and readers to agree on the meaning of the references contained in items~\ref{itm:bg} and \ref{itm:em}.
The FAIR guiding principles, which were designed to help scientists organize data for Findability, Accessibility, Interoperability, and Reusability \citep{Wilkinson_2016}, provide valuable perspective on this goal.

Central to many of the FAIR guidelines is the use of linked semantic data \citep{Bizer_2009}, in which an object to be interpreted is signified by a link to a resource on the World Wide Web, typically referred to as a Uniform Resource Identifier (URI) or a hyperlink.  This link serves as a unique identifier and a resolvable reference to metadata, and the linked content can be  curated by the study authors and other members of the community \citep{Khan_2011}.

%Semantic data are structured by reference to knowledge models called ontologies, which describe the relationships among different types of entities \citep{Madin2008}.  Some preliminary descriptions of the entity types involved in LCA computation, and their relationships to one another, have recently been developed \citep{Ciroth_Srocka_2014, Janowicz_WOP_2015, Kuczenski_JCP_2016}.

Before the goal of easily reproducible foreground models can be realized, this linked data foundation must be laid.  Background database providers should ensure that their users have access to \emph{stable semantic references} for publicly available datasets that denote a particular activity in a particular database version and configuration.  These references should take the form of URIs that can be accessed using any Internet browser and can provide both author and reader with documentary information describing the referred dataset, including version information; database configuration; the available reference flows and their dimensions (quantitative units of measure); and a mechanism for the reader to obtain access to the exchange data or LCI/LCIA computations for the purposes of model validation.  A dependency reference in disclosure item~\ref{itm:bg} should include both an activity and a reference flow.


Similarly, %LCIA method providers should ensure that their methods make use of stable semantic references to elementary flows and contexts.
elementary flows and contexts should be given stable references that are held in commomn across data providers. Current research has revealed widespread challenges to finding consistency on the identities of elementary flows despite the community's longstanding awareness of the problem \citep{Speck_2015,Herrmann_2015}, and several mutually inconsistent reference flow sets now exist \citep{Edelen_2017}.  The conventional understanding of a ``flow'' as comprising a substance and a context together (e.g. methane, emissions to air) causes a combinatorial increase in the number of flows in a database, and also multiplies the potential points of disagreement across sources.  I reiterate Edelen et al.'s (ibid.) suggestion to explicitly regard substances (which can easily include non-material ``flowables'' such as land occupation and transformation) and contexts as independent semantic classes.  In this case, an elementary flow entry in the disclosure item~\ref{itm:em} would require both a flowable and a context.

\endinput

It is necessary for the entries in disclosure items~\ref{itm:bg} and \ref{itm:em} to be unambiguous and easy to interpret.  Entries in \ref{itm:bg}, which represent rows in $A_d$, must clearly identify the originating database, version, and configuration, along with the precise dataset, that dataset's reference flow, and the dimension (reference quantity or unit) of that flow.  In addition, the sign of the numeric entry in $A_d$ must be consistent with the implementation of the background dataset.  Similar requirements also hold for entries in \ref{itm:em}, which represent rows in $B_f$.
