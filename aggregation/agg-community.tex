\subsection{Recommendations for Advancing LCA Practice}

The most transformative aspect of the disclosure framework is its ability to describe the productive output of LCA modeling in a way that transcends differences in software environments and databases.  %In current practice, there is no way to separate the PSM from the computational environment used to create it; however, the formalism allows a study to be put into terms that are software independent.
%The construction and and computation of Eq.~\ref{eqn:study} is equivalent to the operation of all LCA software that employs static background databases.
A product system model described in the disclosure framework can %thus
be implemented in any software system, though the manner of doing so, as well as availability of advanced features for analysis and visualization, varies widely.

To advance LCA practice, the community should embrace the notion of a software-independent PSM description and consider how to best achieve it.  The SETAC North America roadmap module for PSM description and revision \citep{Kuczenski_JLCA_2018} includes milestones toward this goal.  The disclosure framework presented here is offered as a candidate for Milestone 2.1, ``a minimal description of a PSM.''  The community should give this proposal a robust critical evaluation, and researchers should consider how to conceptualize their own models in terms of the framework.  Other milestones and cross-cutting issues are of crucial interest:
\begin{enumerate}[label=(\alph*)]
\item LCI database providers must provide their users with unambiguous, static, resolvable references to background datasets (Milestone 1.1) so that their use in a model may be documented precisely (Milestone 3.1);
\item The transformation of multi-functional processes to single-function processes should be described using the framework (Milestones 1.2 and 1.4);
\item Software makers should enable their users to automatically generate lists of foreground, background, and emission flows a model foreground (disclosure items \ref{itm:fg} through \ref{itm:em}) and to export foreground matrices (disclosure items \ref{itm:af} through \ref{itm:bf}) from a model in a concise format (Milestones 2.2 and 3.4);
\item During ISO critical review, practitioners and reviewers should strive to come to agreement on the formal structure and contents of the model being reviewed;
\item The LCA research community, including interested parties within the International Society for Industrial Ecology, the UNEP/SETAC Life Cycle Initiative, and other organizations, should collectively pursue consistent, shared definitions of quantities of measure, flows, and contexts used in LCIA.  %These definitions should be published and curated using Semantic Web techniques and practices.
LCIA method developers should immediately begin using these consensus definitions instead of their own internal lists when publishing flow characterization data, so that software makers can interpret the publications correctly.  %A characterization of a flow defined privately, into a context defined privately, cannot be interpreted meaningfully by the public.
\end{enumerate}
As these milestones are pursued and achieved, it will be possible for a standalone description of a PSM to be computed and extended by another party, regardless of the software used to construct the model originally.  This will tremendously improve the capacity for critical review and reuse of LCA models.

\endinput


Reproducing a study described in the disclosure framework is equivalent to constructing and computing Eq.~\ref{eqn:study}, which I assert is universally found in LCA computations using static background databases.





A PSM fully defined in concise textual terms can be shared and distributed without 


The disclosure framework forproduct system models can advance LCA practice by transcending differences in modeling across databases and software systems.  


can advance LCA practice by 






\endinput


* Role of ISIE or SETAC group

- unclear
it seems like we would need buy-in before going any farther.
What can people do?


* Responsibility of database and LCIA method providers


* Next steps-- part of standards?


* Potential for distributed LCA computation




The need for a shareable product system model not dependent on any single software system was identified as a key objective for advancing the state of the art in LCA \citep{Kuczenski_JLCA_2018}.  The framework introduced in this article contributes to several milestones in that document, namely the ability to identify what datasets are used in a model and how they are linked together.  The framework can also be used to document how allocation was performed.  \emph{This does not really answer the question}


What is the role of the community?
 - need to band data providers together under a common framework
   = need stble semantic references
   = need provider-independent community-oriented definitions for flows, quantities, and contexts
   = need open source tools 

Milestones:


Questions from the reviewers:

R2: **I suggest to include another brief section 5.5., containing a comment on the role of the ISIE (it’s their journal!) or the SETAC group in facilitating the establishment of science-wide classifications. What about the responsibility of the database and LCIA-method providers?

R3: This would help in emphasizing the next steps, especially if the results draw on discussions within the SETAC working group, and possibly answer the question of whether this could become part of future LCA standards.




Community is responsible for: consensus definitions of flows, quantities, and contexts
Mainly, the community is responsible for disclosing product system models according to this definition
software makers need to enable this capability

``How do we use it?''

It is my contention that any model can be represented within the disclosure framework
so practitioners who are interested in facilitating sharing and/or reuse of their models should implement them in such a framework.
Critical reviewers should consider ISO reports with a thought to how the model would be represented in the disclosure framework, were it to be disclosed
it will obviously take time for software makers to create new functionalities for generating disclosures
EPDs can be accompanied by disclosures
PCRs can be written as PSM specifications
WEB SERVICES can enable direct computation of LCA results from PSM documents (how? entity list includes a mapping to a web service instead of to a static document; queries can be asked of the web service to determine quantitative aspects)


Data providers are responsible for: stable semantic references to data sets
LCIA method providers are responsible for: using community-defined flows instead of their own flows


What is all this good for?

 - we want to better

 ---------

 most important innovation here is separating bg from fg data
 if in fact bg data are static, and become referenceable via the Internet,


 Distributed LCA

 
