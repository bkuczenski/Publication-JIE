\subsection{Advancing LCA Practice}




\endinput

The need for a shareable product system model not dependent on any single software system was identified as a key objective for advancing the state of the art in LCA \citep{Kuczenski_JLCA_2018}.  The framework introduced in this article contributes to several milestones in that document, namely the ability to identify what datasets are used in a model and how they are linked together.  The framework can also be used to document how allocation was performed.  \emph{This does not really answer the question}


What is the role of the community?
 - need to band data providers together under a common framework
   = need stble semantic references
   = need provider-independent community-oriented definitions for flows, quantities, and contexts
   = need open source tools 

Milestones:


Questions from the reviewers:

R2: **I suggest to include another brief section 5.5., containing a comment on the role of the ISIE (it’s their journal!) or the SETAC group in facilitating the establishment of science-wide classifications. What about the responsibility of the database and LCIA-method providers?

R3: This would help in emphasizing the next steps, especially if the results draw on discussions within the SETAC working group, and possibly answer the question of whether this could become part of future LCA standards.




Community is responsible for: consensus definitions of flows, quantities, and contexts
Mainly, the community is responsible for disclosing product system models according to this definition
software makers need to enable this capability

``How do we use it?''

It is my contention that any model can be represented within the disclosure framework
so practitioners who are interested in facilitating sharing and/or reuse of their models should implement them in such a framework.
Critical reviewers should consider ISO reports with a thought to how the model would be represented in the disclosure framework, were it to be disclosed
it will obviously take time for software makers to create new functionalities for generating disclosures
EPDs can be accompanied by disclosures
PCRs can be written as PSM specifications
WEB SERVICES can enable direct computation of LCA results from PSM documents (how? entity list includes a mapping to a web service instead of to a static document; queries can be asked of the web service to determine quantitative aspects)


Data providers are responsible for: stable semantic references to data sets
LCIA method providers are responsible for: using community-defined flows instead of their own flows


What is all this good for?

 - we want to better
