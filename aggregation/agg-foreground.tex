\subsection{Interpreting the Foreground}

The foreground of an LCA study is a collection of nodes that represent points in the inventory model where flows are exchanged.  The foreground matrix $A_f$ describes in precise terms what components are included in the product system and how they connect to one another.  The foreground is modelled as a weighted directed graph in which there is a 1:1 correspondence between nodes and product flows.  The foreground matrix $A_f$ reports the adjacency and weights of edges in the graph, and thus describes the relationships among these nodes.  These nodes typically represent unit processes, but they can also indicate exchanges across an implicit system boundary that encloses the foreground.

A reviewer with access to items \ref{itm:fg} and \ref{itm:af} in the disclosure would be able to construct $A_f$ and automatically create a process-flow diagram, annotated with information about each node.  Generally, the foreground matrix $A_f$ can represent any process-flow model with a 1:1 correspondence between processes and flows, i.e. any allocated supply-and-use table, but some fundamental designs are common.  Some simple foreground models are shown in Figure~\ref{fig:fragments}.  

\begin{figure}[p]
  \begin{center}
    \input{fig3-fragments.fig}
    \caption{Equivalent matrix representations and graphs for different foregrounds.}
    \label{fig:fragments}
  \end{center}
\end{figure}


The first (a) is a sequential model, in which each node requires one foreground input and generates one output. This model is equivalent to a ``gate to gate'' model.  Here the weights $k_i$ indicate the amount of the preceding reference flow that is required by the subsequent node.  Figure~\ref{fig:fragments}(b) shows an additive model, in which the outputs of several foreground nodes are added together, equivalent to a ``mixer'' or a horizontal average.  In this arrangement the weights %represent the relative weights of each input, and
should add up to a unit output of the reference node.  Finally, Figure~\ref{fig:fragments}(c) shows a foreground model with a cyclic dependency, where some of the reference output is consumed by another foreground node.

A typical PSM may contain multiple modules or fragments that are interconnected.  An example of foreground composed of several fragments is illustrated in Figure~\ref{fig:fragments}(d).  Here, the nodes labeled 1--5 represent one fragment, which generates the foreground's canonical reference flow $\tilde{y}$. This fragment requires two interior flows from separate fragments ($y_0$ and $y_1$), and has two unconnected flows (4 and 5).  The reference $y_0$ is supplied by a second fragment, made up of nodes 6-8.  The reference $y_1$ is supplied by another fragment made of only one node (9). The reference flow $y_1$ is consumed in two different places by the other fragments.




