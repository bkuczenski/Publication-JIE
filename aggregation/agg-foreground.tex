\subsection{The Foreground System Boundary and Cut-off Flows}

The foreground of an LCA study is a collection of nodes that represent points in the inventory model where flows are exchanged.  The foreground matrix $A_f$ describes in precise terms what components are included in the product system and how they connect to one another.  The foreground is modelled as a weighted directed graph in which there is a 1:1 correspondence between nodes and product flows.  The foreground matrix $A_f$ reports the adjacency and weights of edges in the graph, and thus describes the relationships among these nodes.  Some simple foregrounds are described in the supporting information.

Generally, the direct requirements matrix can represent any process-flow model with a 1:1 correspondence between processes and flows.  Normally this is understood as a requirement to include only allocated single-output processes.  However, the foreground disclosure framework is also well suited to documenting how multi-output processes are transformed into single-output processes through either partitioning allocation or different forms of system expansion.  Different approaches for creating foregrounds to model multi-output process are presented in the supporting information.

An implicit ``foreground system boundary'' can be imagined to contain all the nodes in $A_f$.  Non-zero values in this matrix represent exchanges inside this boundary.  If a column in $A_f$ is nonempty, then the corresponding node has exchanges with other nodes in the foreground.  On the other hand, if a column in $A_f$ is empty, then that is an indication that the corresponding product flow is crossing the foreground system boundary.  In order for this flow to contribute any impacts, it must either be linked to a background process in the dependency matrix, or represented as an emission. 

A flow whose entire column consists of 0s throughout $A_f$, $A_d$, and $B_f$ is a ``cut-off'' that exits the model with zero burdens.  Specifying cut-off criteria and reporting of cut-off flows is an important part of system boundary definition.  Thus, reviewing cut-off flows (including evaluating their significance) is a crucial part of review that is facilitated by having a structured disclosure of the model.

